\documentclass{article}

% Language setting
% Replace `english' with e.g. `spanish' to change the document language
\usepackage[english]{babel}

% Set page size and margins
% Replace `letterpaper' with`a4paper' for UK/EU standard size
\usepackage[letterpaper,top=2cm,bottom=2cm,left=3cm,right=3cm,marginparwidth=1.75cm]{geometry}

% Useful packages
\usepackage{amsmath}
\usepackage{graphicx}
\usepackage[]{verbatim}
\usepackage[colorlinks=true, allcolors=blue]{hyperref}

% This package is handy for captioning figures, you can set caption style here as well
\usepackage[font={small,it}]{caption}

% This is important for position images as latex will put your image where it best fits unless you tell it otherwise
\usepackage{float}

% https://www.youtube.com/watch?v=WLqVLNzIN9E
% \usepackage[framed, numbered]{matlab-prettifier}
% \lstlistoflistings
% \usepackage{listings}
% \usepackage{xcolor}

\usepackage[]{csharp}

% This allows really nice formatting for MATLAB code, it's the main plug in package that I use
% \usepackage[numbered,framed]{mcode}

\lstset{
  numbers=none, % Prefer not to set numbers due to the inconvience in copying 
}
\title{CAB202 Asynchronous Programming with Async Await
\\\large A survey on its usage in c\# and javascript and the paradigms related}

\author{Anhad Ahuja}

\begin{document}

\maketitle



\section{Introduction}
Async and await is fundamentally syntax to support modern asynchronous programming paradigms.

Imperative or procedural programming is highly sought after (paradigm wise) in the modern software industry. 
Imperative programming offers a more obvious flow of state and progression to the program and is thus known as algorithmic programnming.

This api does not seem to be specifically aimed at performance for reasons explored in the implementation section
however, it does seem ver applicaple to developers that wanted to do the same thing consistently which is to wait for actions that are out of thier control. 
Black box approaches being supported by this feature further aids to the facade paradigm heavily induced within the c\# object oriented ecosystem. 





\newpage

\section{How async works under the hodo in c\#}
\subsection{State Machine implementation}

\subsection{SynchronizationContext}
The context that a piece of code gets run on is
% https://stackoverflow.com/questions/22645024/when-would-i-use-task-yield
% https://docs.microsoft.com/en-us/archive/msdn-magazine/2011/february/msdn-magazine-parallel-computing-it-s-all-about-the-synchronizationcontext

\begin{lstlisting}[language={[Sharp]C}]

      public HomeViewModel()
        {
            SelectedIndex = 0;
            Games = new ObservableCollection<Game>(Db.games);
            DispatcherTimer timer = new DispatcherTimer();
        }
\end{lstlisting}

\section{History}
\subsection{How it was introduced}
\subsection{Previous paradigm: used APM and history of introduction}
This implementation can be awknowledged as using the message passing paradigm

\section{Using lazy evaluation of enumerables: \texttt{IAsyncEnumarable}}
%  await foreach (var stepPoint in step.PlaceStep(sprinklerParameters, region, region, sprinklerRegions, possibleTilePositions, sprinklerLines, sprinklerObstructions, sprinklerPoints.Values.ToArray(), true, svgWriter, cancellationToken))
% https://docs.microsoft.com/en-us/archive/msdn-magazine/2019/november/csharp-iterating-with-async-enumerables-in-csharp-8
% https://stackoverflow.com/questions/25009437/running-multiple-async-tasks-and-waiting-for-them-all-to-complete

\section{UI based example and how to avoid system locking}
Locks up Ui, conventionally heavy computationaly or IO based operation that locks up UI thread is undesirable.
It could be possible to create a new thread using system API and delegate tasks to that,
but that has a lot of computational and programmer overhead.
Generally not a trivial matter especially to sync up content and data.


\section{\texttt{Task.Run} based example}


\section{Additional quirks and useful features of Task based await}
\subsection{Cancellation tokens}
\subsection{\texttt{IProgress}}


\section{Comparison to javascript}

\section{Javascript similiar example using \texttt{Promise}}


\bibliographystyle{alpha}
\bibliography{sample}

\end{document}